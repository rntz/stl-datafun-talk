\documentclass[xcolor=table,usenames,dvipsnames,svgnames]{beamer}

%% %% apparently this magic helps avoid the dreaded
%% %% ``Too many math alphabets used in version normal''
%% %% error. Yuk.
%% \newcommand\hmmax{0}
%% \newcommand\bmmax{0}
%% %% end magic

\usepackage{amssymb,amsmath,amsthm,latexsym}
\usepackage{url,hyperref}
\usepackage{mathpartir}         % \mathpar, \infer
\usepackage{booktabs}           % \midrule
\usepackage{nccmath}            % fix spacing

% \usepackage{mathtools} % \dblcolon
% \usepackage{accents}   % \underaccent
% \usepackage{stmaryrd}  % \llbracket,\rrbracket
% \usepackage{proof}     % \infer
% \usepackage{graphicx}  % \scalebox, \rotatebox

% TODO: frame titles in small caps
\usefonttheme{professionalfonts}
\renewcommand{\familydefault}{\rmdefault}
\usepackage[scaled=0.98]{XCharter}
%\usepackage[scaled=0.88]{librebaskerville} % .855
\usepackage[semibold,scaled=0.968]{sourcesanspro}
\usepackage[scaled=1.0328,varqu,var0]{inconsolata}
%\usepackage[libertine,scaled=1.086]{newtxmath}
\usepackage{eulervm}\makeatletter\edef\zeu@Scale{1.0341}\makeatother
\usepackage[bb=boondox]{mathalfa} % or bb=ams
\usepackage{anyfontsize}
\usepackage[T1]{fontenc}


%% Commands
\newcommand{\N}{\mathbb{N}}
\newcommand{\x}{\times}

%% TODO: remove this.
\newcommand{\df}{\color{RoyalBlue}}
\renewcommand{\df}{}

\newcommand{\GD}{\Delta}


%% Metadata
\title{Datafun}
\subtitle{a functional query language}
%% FIXME: causes shittons of warnings argh.
\author{\texorpdfstring{Michael Arntzenius\newline
  \href{mailto:daekharel@gmail.com}{daekharel@gmail.com}\newline
  \url{http://www.rntz.net/datafun}}{Michael Arntzenius}}
\date{Curry On, July 2018}

\begin{document}

\maketitle


\begin{frame}\centering\LARGE\bf Early-stage research \end{frame}

\begin{frame}\LARGE \centering
  \textbf{What if \textcolor{DarkOrchid}{programming languages}\\
    were more like \textcolor{DarkOrchid}{query languages}?}

  %% \vspace{2em}

  %% \textbf{What if \textcolor{DarkOrchid}{query languages} were more\\
  %%   like \textcolor{DarkOrchid}{programming languages}?}
\end{frame}


\begin{frame}\Large
  \begin{enumerate}
    \itemsep 1.5em
  \item {\bf What's a functional query language?}
  \item From Datalog to Datafun
  \item Incremental Datafun
  \end{enumerate}
\end{frame}


\begin{frame}{SQL}\large
  \begin{minipage}{0.4\textwidth}
    \centering
    \begin{tabular}{ll}
      \textbf{Parent} & \textbf{Child}\\\midrule
      Arathorn & Aragorn\\
      Drogo & Frodo\\
      E\"arwen & Galadriel\\
      Finarfin & Galadriel\\
      \hfill\vdots & \hfill\vdots
    \end{tabular}
  \end{minipage}
  %
  %% \pause
  \hfill
  %
  \begin{minipage}{0.53\textwidth}
    %% TODO: is ``parentage'' the best name?
    \texttt{SELECT parent\\FROM parentage\\WHERE child = "Galadriel"}
  \end{minipage}
\end{frame}


\begin{frame}{Tables as sets}\large
  \begin{minipage}{0.4\textwidth}
    \centering
    \begin{tabular}{ll}
      \textbf{Parent} & \textbf{Child}\\\midrule
      Arathorn & Aragorn\\
      Drogo & Frodo\\
      E\"arwen & Galadriel\\
      Finarfin & Galadriel\\
      \hfill\vdots & \hfill\vdots
    \end{tabular}

    %% \vspace{1.5em}{\large \uncover<2>{\textsc{as a table}}}

  \end{minipage}
  %
  \hfill{\LARGE =}\hfill
  %
  \begin{minipage}{0.53\textwidth}
    \centering

    \begin{tabular}{l}
      \color{gray}{// set of (parent, child) pairs}\\
      \{(Arathorn, Aragorn),\\
      \phantom{\{}(Drogo, Frodo),\\
      \phantom{\{}(E\"arwen, Galadriel),\\
      \phantom{\{}(Finarfin, Galadriel),\\
      \phantom{\{}...\,\} \phantom{\vdots}
    \end{tabular}

    %% \vspace{1.5em}{\large \textsc{as a relation}}
  \end{minipage}

\end{frame}


\begin{frame}\Large\centering
  \bf Tuples and sets are just datatypes!

  \vspace{2em}\pause

  If tables are sets, what are queries?
\end{frame}


\begin{frame}{Queries as set comprehensions}\Large\centering
  \begin{minipage}{0.7\textwidth}
    \texttt{SELECT {\color{orange}parent}\\
      FROM {\color{Green}parentage}\\
      WHERE {\color{DarkOrchid}child = "Galadriel"}}
  \end{minipage}

  \pause
  \vspace{1.5em}
  {\LARGE $\Longrightarrow$}
  \vspace{1em}

  \begin{minipage}{1.0\textwidth}
    \df\tt \{ {\color{orange}parent}
    | {\color{Green}(parent, child) in parentage}\\
    \phantom{\{ parent }, {\color{DarkOrchid}child = "Galadriel"} \}
  \end{minipage}
\end{frame}

\begin{frame}{Queries as set comprehensions: finding siblings}\large\centering
  \begin{minipage}{0.9\textwidth}
    \tt SELECT DISTINCT {\color{orange}A.child, B.child}\\
    FROM {\color{Green}parentage A} INNER JOIN {\color{RoyalBlue}parentage B}\\
    ON {\alt<2->{\color{red}}{}A.parent = B.parent}
    \\WHERE {\color{DarkOrchid}A.child <> B.child}
  \end{minipage}

  \vspace{1.5em}
  {\LARGE $\Longrightarrow$}
  \vspace{1em}

  \begin{minipage}{0.8\textwidth}
    \df\tt\{ {\color{orange}(a,b)}
    | {\color{Green}({\alt<2->{\color{red}}{}parent}, a) in parentage}\\
    \phantom{\{ (a,b) }, {\color{RoyalBlue}({\alt<2->{\color{red}}{}parent}, b) in parentage}\\
    \phantom{\{ (a,b) }, {\color{DarkOrchid}not (a = b)} \}
  \end{minipage}
\end{frame}


%% \begin{frame}{Queries as set comprehensions: grandparenting yourself}\large\centering
%%   \begin{minipage}{0.9\textwidth}
%%     \tt SELECT parent\\
%%     FROM parentage A INNER JOIN parentage B\\
%%     ON A.child = B.parent AND A.parent = B.child
%%   \end{minipage}

%%   \vspace{1.5em}
%%   {\LARGE $\Longrightarrow$}
%%   \vspace{1em}

%%   \begin{minipage}{0.8\textwidth}
%%     \df\tt\{ a | (a, b) in parentage\\
%%     \phantom{\{ a }, (!b, !a) in parentage \}
%%   \end{minipage}
%% \end{frame}


\begin{frame}{Recipe for a functional query language}\Large
  %% \textbf{Recipe for a functional query language:}\vspace{0.5em}

  \begin{enumerate}\itemsep 1.5em
  \item Take a functional language
  \item Add sets and set comprehensions
  \item {\color{red} ... done?}
  \end{enumerate}
\end{frame}


\begin{frame}\centering{\bf\huge But can it go fast?}\end{frame}

\newcommand{\inftyloop}{\ensuremath{\infty\text{-loop}}}
\newcommand{\elemexp}{{\alt<3>{{\color{red}print x}}{...}}}
\newcommand{\aloop}{x in \alt<3-4>{\{\alt<3>{"hello"}{}\}}{EXPR1}}
\newcommand{\bloop}{{\color{DarkOrchid}%
y in \alt<3>{\{0,1\}}{\alt<4>{\color{red}\inftyloop}{EXPR2}}}%
}

\begin{frame}{Loop reordering}
  \large\centering
  \texttt{\{ \elemexp~| \aloop{}, \bloop{} \}}
  \alt<4>{{$~\Longrightarrow~$\{\}}\phantom{-loo}\,}{}
  %
  \vspace{0.5em}\\
  {\Large \alt<1>{=?}{\alt<2>{\color{red}}{}$\neq$}}
  %
  \vspace{0.5em}\\
  \texttt{\{ \elemexp~| \bloop{}, \aloop{} \}}
  \alt<4>{{\color{red}$~\Longrightarrow~\inftyloop$}}{}

  \pause
  \vspace{3em}\Large
  \begin{minipage}{0.5\textwidth}
    \begin{enumerate}
    \item {\alt<3>{\color{red}}{} Side-effects}
    \item {\alt<4>{\color{red}}{} Nontermination}
    \end{enumerate}
  \end{minipage}
\end{frame}


\begin{frame}{Recipe for a functional query language, v2}\Large
  \begin{enumerate}\itemsep 1.5em
  \item Take a \textbf{pure, total} functional language
  \item Add sets and set comprehensions
  \item {\bf Optimize!}
  \item {\color{red} ... done?}
  \end{enumerate}
\end{frame}


\begin{frame}\centering\huge\bfseries But what about closures?\end{frame}

\begin{frame}\Large
  \emph{``A Practical Theory of Language-Integrated Query''}
  \\[.2em]\large
  [Cheney, Lindley, \& Wadler, ICFP 2013]\\[.1em]
  sez:

  \vspace{.5em}
  \centering\Large {\bfseries Just normalize} (i.e.\ \textbf{inline everything})\textbf{!}*

  \vspace{.5em}\large
  *Requires totality.
\end{frame}


\begin{frame}{Recipe for a functional query language, v3}\Large
  \begin{enumerate}\itemsep 1.5em
  \item Take a pure, total functional language
  \item Add sets and set comprehensions
  \item {\bfseries Normalize} away higher-order functions?
  \item Optimize!
  \item {\color{red} ... done?}
  \end{enumerate}
\end{frame}

\begin{frame}
  \centering\huge\bfseries How do we execute queries efficiently, anyway?
\end{frame}

\begin{frame}
  \Large
  \emph{``Leapfrog Triejoin: a simple, worst-case optimal join algorithm''}
  \\[.2em]\large
  [Todd Veldhuizen, arXiv 2013]

  \pause\vspace{1em}
  \huge\centering\color{red} Patented! :(
\end{frame}


\begin{frame}{}\large\parskip 0.4em
  \textsc{What have we gained?}
  \begin{itemize}\itemsep 0.5em
  \item Can factor out repeated patterns with\\ \textbf{higher-order functions}
  \item Sets are \textbf{just ordinary values}
    %% we can put them inside other sets
    %% we can stick them in a closure
  \item Sets, bags, lists: choose your container semantics!
  \end{itemize}

  \vspace{1em}\pause

  \textsc{At what cost?}
  \begin{itemize}
  \item \textbf{Implementation complexity}:\\
    {\small GC, closures, nested sets, optimizing comprehensions...}
  \item \textbf{Re-inventing the wheel}:\\
    {\small persistence, transactions, replication...}
  \end{itemize}

  %% \textsc{The drawback:} \textbf{Implementation complexity!}\pause
  %% \begin{itemize}\itemsep 0.5em
  %% \item Garbage collection
  %% \item Closures
  %% \item Nested sets
  %%   %% re-implementing existing features
  %% \item \color{red}\bf Transactions?!
  %% \end{itemize}
\end{frame}

%% TODO: slide about pros and cons / what we gained, what we lose
%% TODO: moar examples of Datafun


\begin{frame}\Large
  \begin{enumerate} \itemsep 1.5em \color{gray}
  \item What's a functional query language?
  \item {\color{black}\bf From Datalog to Datafun}
  \item Incremental Datafun
  \end{enumerate}
\end{frame}


\begin{frame} \large \centering

  \begin{tabular}{l@{\hskip 1.5em}l}
    \textbf{Parent} & \textbf{Child}\\\midrule
    Arathorn & Aragorn\\
    Drogo & Frodo\\
    E\"arwen & Galadriel\\
    Finarfin & Galadriel\\
    \hfill\vdots & \hfill\vdots
  \end{tabular}

  \vspace{2em}

  %% robsimmons: ``It's worth having languages where the
  %% important ideas are foundational to the language.''
  %% use this?
  {\Large \bf Is E\"arendil one of Aragorn's ancestors?}

\end{frame}


\newcommand{\atom}[1]{\textcolor{brown}{#1}}
\begin{frame}{Datalog in a nutshell}
  \large
  \atom{\alt<2-3>{ancestor($X$,$Z$)}{$X$ is $Z$'s ancestor}}
  \alt<3>{:-}{if}
  \atom{\alt<2-3>{parent($X$, $Z$)}{$X$ is $Z$'s parent}}.
  \\\vspace{1em}
  \atom{\alt<2-3>{ancestor($X$, $Z$)}{$X$ is $Z$'s ancestor}}
  \alt<3>{:-}{if}
  \atom{\alt<2-3>{parent($X$, $Y$)}{$X$ is $Y$'s parent}}\alt<3>{,}{ and}
  \atom{\alt<2-3>{ancestor($Y$, $Z$)}{$Y$ is $Z$'s ancestor}}.

  %% \vspace{2em}
  %% \uncover<3>{\bf This is valid Datalog.}
\end{frame}

%% \begin{frame}\LARGE\centering
%%   \textbf{How can we do this\\ in a functional query language?}
%% \end{frame}


\begin{frame}\Large
  Datalog is \textbf{deductive}: it chases rules to their logical conclusions.
  \\\vspace{2em}
  Can we capture this feature \textbf{functionally}?
\end{frame}


\newcommand{\conc}{\color{Green}}
\newcommand{\fires}{\bfseries}
\newcommand{\premisea}{\color{Blue}}
\newcommand{\premiseb}{\color{Rhodamine}}
\begin{frame}{}

  {
  \textbf{Procedure:}
  \begin{enumerate}\itemsep 0em
  \item Pick a rule.
  \item Find facts satisfying its premises.
  \item Add its conclusion to the known facts.
  \end{enumerate}}

  \textbf{Rules:}
  \begin{quote}\sf
    {\alt<2-4,6>{\fires}{}
      {\alt<4,6>{\conc}{}ancestor(X,Z)} :-
      {\alt<3-4,6>{\premisea{}}{}parent(X,Z)}.}\\
    {\alt<8>{\fires}{}
      {\alt<8>{\conc}{}ancestor(X,Z)} :-
      {\alt<8>{\premisea}{}parent(X,Y)},
      {\alt<8>{\premiseb}{}ancestor(Y,Z)}.}\\
  \end{quote}

  \textbf{Facts:}
  \begin{quote}\sf
    {\alt<3-4,8>{\premisea}{} parent(Idril, E\"arendil).}\\
    {\alt<6>{\premisea}{} parent(E\"arendil, Elros).}\\
    \uncover<4->
    {\alt<4>{\conc}{}ancestor(Idril, E\"arendil).\quad \alt<4>{\it(new!)}{}}\\
    \uncover<6->
    {\alt<6>{\conc}{}\alt<8>{\premiseb}{}ancestor(E\"arendil, Elros).\quad\alt<6>{\it(new!)}{}}\\
    \uncover<8->
    {\alt<8>{\conc}{}{ancestor(Idril, Elros).\quad \alt<8>{\it(new!)}{}}}
    \alt<9>{}{}
  \end{quote}
\end{frame}


\begin{frame}\Large
  Repeatedly apply a \textbf{\alt<2->{\alt<2>{\color{blue}}{}function}{set of rules}}\\
  until \textbf{\alt<3->{\alt<3>{\color{blue}}{}its output equals its input}{nothing changes}}\\
  \uncover<4->{i.e. it reaches a \textbf{\color{blue}fixed point}}{}

  \vspace{2em}

  \uncover<5->{\LARGE\centering\df fix x = {\it ... function of x ...}}
\end{frame}


\begin{frame}
  \Large

  \textcolor{gray}{\it // Datalog}\\
  {\alt<2>{\bf}{}ancestor(X,Z) :- parent(X,Z).}\\
  {\alt<3>{\bf}{}ancestor(X,Z) :- parent(X,Y), ancestor(Y,Z).}

  \vspace{2em}\df

  \textcolor{gray}{\it // Datafun}\\
  fix ancestor = {\alt<2>{\bf}{}parent}\newline
  \phantom{fix ancestor}
  %$\cup$ (parent $\bullet$ ancestor)
  $\cup$ \alt<3>{\bf}{}\{(x,z) $\mid$ (x,y) in parent\newline
  \phantom{\rm fix ancestor $\cup$ }\phantom{\{(x,z)}, (y,z) in ancestor\}
\end{frame}


\begin{frame}\large
  \textbf{Repeatedly applying:}
  \begin{center}
    ${\color{blue}X} \longmapsto$ parent $\cup$ \{(x,z) $\mid$ (x,y) in parent, (y,z) in {\color{blue}X}\}
  \end{center}

  \parskip 0.5em
  \textbf{Where} parent = \{(Idril, E\"arendil), (E\"arendil, Elros)\}

  \textbf{Steps}:\\
  \def\arraystretch{1.3}
  \begin{tabular}{rl}
    & \alt<1-2>{\color{blue}}{}$\emptyset$
    \\\pause
    $\longmapsto$ & parent $\cup$ \{(x,z) $\mid$ (x,y) in parent, (y,z) in {\alt<1-2>{\color{blue}}{}$\emptyset$}\}
    \\\pause
    =& \alt<4>{\color{blue}}{}parent
    \\\pause
    $\longmapsto$ & parent $\cup$ \{(x,z) $\mid$ (x,y) in parent, (y,z) in {\alt<4>{\color{blue}}{}parent}\}
    \\\pause
    =& \{(Idril, E\"arendil), (E\"arendil, Elros), (Idril, Elros)\}
  \end{tabular}
\end{frame}


\begin{frame}\centering{\bf\huge But can it go fast?}\end{frame}
% And I'm afraid the only answer I can give to this question is
% ``I don't know yet.''
%
% But I am hopeful. And in the next section I'm going to show you one of the
% reasons I'm hopeful.


\begin{frame}\Large
  \begin{enumerate}\color{gray}\itemsep 1.5em
  \item What's a functional query language?
  \item From Datalog to Datafun
  \item {\color{black}\bf Incremental Datafun}
  \end{enumerate}
\end{frame}


\begin{frame}{Three problems}\large
  \begin{enumerate}\itemsep 1em
    \alt<4>{\color{gray}}{}
  \item {\bf View maintenance:} \\
    {\normalsize How do we update a cached query efficiently after a mutation?}\
    \pause
  \item {\bf Semina\"ive evaluation in Datalog:}\\
    {\normalsize How do we avoid re-deducing facts we already know?}
    \pause
  \item \color{black}{\bf Incremental computation:}\\
    {\normalsize How do we efficiently recompute a function as its
      inputs change?}
  \end{enumerate}
\end{frame}

\begin{frame}
  \Large {\itshape ``A Theory of Changes for Higher-Order Languages:
    Incrementalizing $\lambda$-calculi by Static Differentiation''}%
  \vspace{0.1em}\newline\normalsize [PLDI 2014]
  \\\vspace{0.5em}
  \small by Yufei Cai, Paolo G Giarrusso, Tillmann Rendel, and Klaus Ostermann
\end{frame}


\begin{frame}{Static differentiation}
  \vspace{-1em}
  Every type $A$ has a \emph{type of changes}, $\color{blue}\Delta A$.
  \pause
  \begin{align*}
    \GD \mathbb{N} &= \mathbb{Z}\\
    \GD (A \x B) &= \GD A \x \GD B
  \end{align*}

  \pause Every type also gets an operator $\color{blue}\oplus_A : A \to \GD A \to A$.
  \pause
  \begin{align*}
    x \oplus_\N dx &= x + dx\\
    (x,y) \oplus_{A \x B} (dx,dy) &= (x \oplus_A dx, y \oplus_B dy)
  \end{align*}

  \pause A function $f : A \to B$ gets a \emph{derivative},
  $\color{blue} \delta f : A \to \Delta A \to \Delta B$.
  \pause
  \[\setlength{\arraycolsep}{.25em}\begin{array}{rclcl}
    f(x) &=& x^2\\
    \delta f(x)(dx) &=& 2x\cdot dx+dx^2\\
    \pause
    f(x) + \delta f(x)(dx) &=& x^2 + 2x\cdot dx + dx^2 &=& (x + dx)^2
  \end{array}\]

  %% If $M$ uses variables $x, y, z$,\\
  %% then $\color{blue}\delta M$ will use $x, y, z, \color{blue} dx, dy, dz$.

  %% $\color{blue} \delta M$ computes the \emph{change to $M$}\\
  %% resulting from the changes $\color{blue} dx, dy, dz$ to its variables.

  %% \begin{eqnarray*}
  %% \Delta(A \x B) &=& \Delta A \x \Delta B\\
  %% \Delta(A \to B) &=& A \to \Delta A \to \Delta B
  %% \end{eqnarray*}
\end{frame}


\begin{frame}\centering\LARGE
  \textbf{We've extended this to Datafun!}

\end{frame}


%% TODO:
\begin{frame}{Finding fixed points faster with derivatives}
  \large \parskip 1em
  The na\"ive way to find fixed points looks like this:
  $${\emptyset \mapsto f(\emptyset) \mapsto f^2(\emptyset) \mapsto f^3(\emptyset) \mapsto ...}$$
  \pause
  $f^i(\emptyset)$ and $f^{i+1}(\emptyset)$ \textbf{overlap a lot.}

  Computing $f^{i+1}(\emptyset)$ from $f^i(\emptyset)$ \textbf{does a lot of recomputation.}

  \pause
  What if we could only compute \textbf{what changed} between iterations?
\end{frame}

\begin{frame}\Large
  $$\begin{array}{lcl@{\hspace{2em}}lcl}
    x_0 &=& \emptyset
    & dx_0 &=& f(\emptyset)\\
    x_{i+1} &=& x_i \cup dx_i
    & dx_{i+1} &=& \delta f (x_i)(dx_i)\\
  \end{array}$$

  \vspace{1em}
  \textbf{Theorem:} $x_i = f^i(x)$
\end{frame}


%% conclusion
\begin{frame}{Takeaways}\large
  \begin{enumerate}\itemsep 1em
  \item Set comprehensions = queries
  \item Purity \& totality = fearless optimisation
  \item Fixed points = recursive queries {\itshape\small (\`a la Datalog)}
  \item Incremental computation = faster fixed points
  %% \item Datafun has all three!*
  \end{enumerate}

  %% \vspace{1em}\small
  %% * In theory.
\end{frame}

\begin{frame}\Large
  \textbf{Michael Arntzenius}\\
  \href{mailto:daekharel@gmail.com}{daekharel@gmail.com}\\
  \texttt{@arntzenius}

  \vspace{1em}
  \centering \huge \url{rntz.net/datafun}
\end{frame}

\end{document}
