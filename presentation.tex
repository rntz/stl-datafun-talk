\documentclass[xcolor=table]{beamer}

%% apparently this magic helps avoid the dreaded
%% ``Too many math alphabets used in version normal''
%% error. Yuk.
\newcommand\hmmax{0}
\newcommand\bmmax{0}
%% end magic

\usepackage{amssymb,amsmath,amsthm}
\usepackage{latexsym}

\usepackage{array}

\usepackage{url}
\usepackage{hyperref}

%% uses the AMS Euler math font.
\usepackage{euler}

%% for coloneqq
%% \usepackage{mathtools}

%% %% for \underaccent
%% \usepackage{accents}

% for semantic brackets
%% \usepackage{stmaryrd}

%% % for inference rules
%% \usepackage{proof}

%% %% for mathpar environment
%% \usepackage{mathpartir}

% for \scalebox, \rotatebox
%% \usepackage{graphicx}

%% %% for drawing Hasse diagram
%% \usepackage{tikz}

%% %% for censoring things
%% \usepackage{censor}


%% configuration
%% I hate everything
\newcommand{\fixcolors}{\rowcolors{2}{white}{}}
\fixcolors


%% Commands
\newcommand{\N}{\mathbb{N}}
\newcommand{\x}{\times}

%% wherever you see this color, it means runnable Datafun code.
\newcommand{\datafuncolor}{\color{blue}}


%% Metadata
\title{Datafun}
\subtitle{a functional query language}
\author{Michael Arntzenius
  \newline\href{mailto:daekharel@gmail.com}{daekharel@gmail.com}
  \newline\url{http://www.rntz.net/datafun}}
%% TODO:
%% - http://www.rntz.net
%% - daekharel@gmail.com
\date{Strange Loop, September 2017}

\begin{document}

\maketitle


\begin{frame}\LARGE
  %% TODO: grab pithier text from abstract
  \center \textbf{What if programming languages\\
    were more like query languages?}
\end{frame}

\begin{frame}\Large
  \frametitle{Structure of this talk}
  \begin{enumerate}
    \itemsep 1.5em
  \item What's a functional query language?
  \item Datafun's special sauce: fixed points
  \item Incremental computation
  \end{enumerate}
\end{frame}


%% \begin{frame}
%%   \frametitle{The tabular view}
%%   \large

%%   \begin{enumerate}\itemsep 2em
%%   \item Data is represented by \emph{tables}.
%%     %% \\{\small i.e. sets of tuples (rows) with named fields (columns).}

%%   \item Queries are composed from primitive operators on tables.\\
%%     {\small (select, join, union, ...)}
%%   \end{enumerate}

%%   %% \begin{itemize}\itemsep 1em
%%   %% \item Data is represented by \emph{tables}

%%   %% \item Tables are sets of tuples (rows) with named fields (columns)

%%   %% \item Queries \emph{transform tables into other tables}\\

%%   %% \item Queries are built from \emph{query operators}
%%   %% \end{itemize}

%% \end{frame}



\begin{frame}
  \frametitle{SQL}
  %% \centering

  %% hm. what do I talk about here?
  %% data as tables? data as relations?

  \begin{minipage}{0.4\textwidth}
    \centering
    \begin{tabular}{l|l}
      \textbf{Parent} & \textbf{Child}\\\hline
      Arathorn & Aragorn\\
      Drogo & Frodo\\
      E\"arwen & Galadriel\\
      Finarfin & Galadriel\\
      \rowcolor{white}
      \hfill\vdots & \hfill\vdots
    \end{tabular}
  \end{minipage}
  %
  %% \pause
  \hfill
  %
  \begin{minipage}{0.5\textwidth}
    \texttt{SELECT parent\\FROM parentage\\WHERE child = "Galadriel"}
  \end{minipage}
\end{frame}


\begin{frame}
  \frametitle{Tables as sets}
  %% \centering

  %% hm. what do I talk about here?
  %% data as tables? data as relations?

  \begin{minipage}{0.43\textwidth}
    \centering


    \begin{tabular}{l|l}
      \textbf{Parent} & \textbf{Child}\\\hline
      Arathorn & Aragorn\\
      Drogo & Frodo\\
      E\"arwen & Galadriel\\
      Finarfin & Galadriel\\
      \rowcolor{white}
      \hfill\vdots & \hfill\vdots
    \end{tabular}

    %% \vspace{1.5em}{\large \uncover<2>{\textsc{as a table}}}

  \end{minipage}
  %
  \hfill{\LARGE =}\hfill
  %
  \begin{minipage}{0.43\textwidth}
    \centering

    \rowcolors{1}{}{}
    \begin{tabular}{l}
      \color{gray}{// set of (parent, child) pairs}\\
      \{\hspace{1pt}(Arathorn, Aragorn)\\
      , (Drogo, Frodo)\\
      , (E\"arwen, Galadriel)\\
      , (Finarfin, Galadriel)\\
      \,...\,\} \phantom{\vdots}
    \end{tabular}

    %% \vspace{1.5em}{\large \textsc{as a relation}}
  \end{minipage}

\end{frame}


\begin{frame}
  \frametitle{Queries as set comprehensions}
  \centering\Large

  \begin{minipage}{0.8\textwidth}
    %% TODO: is ``parentage'' the best name?
    \texttt{SELECT parent\\FROM parentage\\WHERE child = "Galadriel"}
  \end{minipage}

  \vspace{1.5em}
  {\LARGE $\Longrightarrow$}
  \vspace{1em}

  \begin{minipage}{0.8\textwidth}
    \datafuncolor
    \tt \{ parent \\
    | (parent, child) in parentage\\
    , child == "Galadriel" \}
  \end{minipage}

  %% TODO: more examples
\end{frame}


\begin{frame}
  \frametitle{Recipe for a functional query language}
  \Large
  \begin{enumerate}\itemsep 1.5em
  \item Take a functional language
  \item Add sets and set comprehensions
  \item {\color{red} ... done?}
  \end{enumerate}
\end{frame}


\begin{frame}
  \centering
  {\bf\huge Can it go fast?}
\end{frame}

\begin{frame}
  \frametitle{Loop reordering}
  \large
  \centering

  \texttt{\{ (x,y) | x in \alt<4>{{\color{red}\{\}}}{EXPR1},
    y in \alt<4>{{\color{red}$\infty$-loop}}{EXPR2}\alt<4>{}{,
    \alt<3>{{\color{red}print x}}{...}}\}}
  \alt<4>{{\color{blue}$~\Longrightarrow~$\{\}}\phantom{-loo}\,}{}
  %
  \vspace{0.5em}\\
  {\Large \alt<1>{=?}{\color{red}$\neq$}}
  %
  \vspace{0.5em}\\
  \texttt{\{ (x,y) | y in \alt<4>{{\color{red}$\infty$-loop}}{EXPR2},
    x in \alt<4>{{\color{red}\{\}}}{EXPR1}\alt<4>{}{,
    \alt<3>{{\color{red}print x}}{...}}\}}
  \alt<4>{{\color{blue}$~\Longrightarrow~\infty$-loop}}{}

  \pause
  \vspace{3em}\Large
  \begin{minipage}{0.5\textwidth}
    \begin{enumerate}
    \item {\alt<3>{\color{red}}{} Side-effects}
    \item {\alt<4>{\color{red}}{} Nontermination}
    \end{enumerate}
  \end{minipage}
\end{frame}


\begin{frame}
  \frametitle{Recipe for a functional query language, v2}
  \Large
  \begin{enumerate}\itemsep 1.5em
  \item Take a \textbf{pure, total} functional language
  \item Add sets and set comprehensions
  \item {\color{orange} Optimize.}
  \end{enumerate}
\end{frame}


\end{document}
